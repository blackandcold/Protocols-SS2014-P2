% -*- TeX:de -*-
\NeedsTeXFormat{LaTeX2e}
\documentclass[12pt,a4paper]{article}
\usepackage[german]{babel} % german text
\usepackage[DIV12]{typearea} % size of printable area
\usepackage[T1]{fontenc} % font encoding
%\usepackage[latin1]{inputenc} % most likely on Windows
\usepackage[utf8]{inputenc} % probably on Linux
\usepackage{multicol}

% PLOTTING
\usepackage{pgfplots} 
\usepackage{pgfplotstable}
\usepackage{url}
\usepackage{graphicx} % to include images
\usepackage{tikz}
\usepackage{subfigure} % for creating subfigures
\usepackage{amsmath} % a bunch of symbols
\usepackage{amssymb} % even more symbols
\usepackage{booktabs} % pretty tables
\usepackage{makecell} % multi row table heading

% a floating environment for circuits
\usepackage{float}
\usepackage{caption}

%\newfloat{circuit}{tbph}{circuits}
%\floatname{circuit}{Schaltplan}

% a floating environment for diagrams
%\newfloat{diagram}{tbph}{diagrams}
%\floatname{diagram}{Diagramm}

\selectlanguage{german} % use german

\begin{document}

%%%%%%% DECKBLATT %%%%%%%
\thispagestyle{empty}
			\begin{center}
			\Large{Fakultät für Physik}\\
			\end{center}
\begin{verbatim}


\end{verbatim}
							%Eintrag des Wintersemesters
			\begin{center}
			\textbf{\LARGE SS 14}
			\end{center}
\begin{verbatim}


\end{verbatim}
			\begin{center}
			\textbf{\LARGE{Physikalisches Praktikum\\ für das Bachelorstudium}}
			\end{center}
\begin{verbatim}




\end{verbatim}

			\begin{center}
			\textbf{\LARGE{PROTOKOLL}}
			\end{center}
			
\begin{verbatim}

\end{verbatim}

			\begin{flushleft}
			\textbf{\Large{Experiment (Nr., Titel):}}\\
							%Experiment Nr. und Titel statt den Punkten eintragen
			\LARGE{PS06 Strahlung}	
			\end{flushleft}

\begin{verbatim}

\end{verbatim}	
							%Eintragen des Abgabedatums, oder des Erstelldatums des Protokolls
			\begin{flushleft}
			\textbf{\Large{Datum:}} \Large{20.3.2014}
			\end{flushleft}
			
\begin{verbatim}
\end{verbatim}
							%Namen der Protokollschreiber
		\begin{flushleft}
			\textbf{\Large{Namen:}} \Large{Patrick Braun, Johannes Kurz}
			\end{flushleft}

\begin{verbatim}


\end{verbatim}
							%Kurstag und Gruppennummer, zb. Fr/5
			\begin{flushleft}
			\textbf{\Large{Kurstag/Gruppe:}} \Large{DO/4}
			\end{flushleft}

\begin{verbatim}

\end{verbatim}
							%Name des Betreuers, das Praktikum betreute.
			\begin{flushleft}
			\LARGE{\textbf{Betreuer:}}	\Large{Erhard Schafler}	
			\end{flushleft}

%%%%%%% DECKBLATT ENDE %%%%%%%
\pagebreak
\setlength{\columnsep}{20pt}
\begin{multicols}{2}

%%%%%%%%%%%%%%%%%%%%%%%%%%%%%%%%%%%%%%%%%%%%%%%%

%\begin{figure}[H]
%	\centering
%	\includegraphics[scale=0.35]{./data/beugung.png}
%	\caption{Beugungsmuster Einzelspalt (echtes Foto; schwarz durch weiß ersetzt)}
%	\label{fig:beugungsmuster}
%\end{figure}


%\begin{figure}[H]
%	\centering
%	\pgfplotstabletypeset[
%			columns={abstand, n},
%			col sep=&,
%			columns/abstand/.style={precision=2, zerofill, column name=\makecell{$Abstand$\\$(\pm 0.05)[mm]$} }, 
%			columns/n/.style={column name=\makecell{$n$\\$(Ordnung)$}, precision=0},
%			every head row/.style={before row=\hline,after row=\hline\hline},
%			every last row/.style={after row=\hline},
%			every first column/.style={column type/.add={|}{} },
%			every last column/.style={column type/.add={}{|} }
%			]{
%			abstand & n
%			12.9 & 1
%			24.45 & 2
%			37.40 & 3
%			49.35& 4
%			62.45 & 5
%			74.45 & 6
%			87.45 & 7
%			100.25 & 8
%			
%			}
%	\caption{Messwerte Einzelspalt}
%	\label{tab:werte_einzelspalt}
%\end{figure}



%%%%%%%%%%%%%%%%%%%%%%%%%%%%%%%%%%%%%%%%%%%%%%%%
%%%%%%%%%%%%%%%%%%%%%%%%%%%%%%%%%%%%%%%%%%%%%%%%
\section{Planck'sches Wirkungsquantum}

\subsection{Grundlagen, Theorie und Versuchsaufbau}


\subsection{Resultate}

\subsection{Diskussion}

%%%%%%%%%%%%%%%%%%%%%%%%%%%%%%%%%%%%%%%%%%%%%%%%
%%%%%%%%%%%%%%%%%%%%%%%%%%%%%%%%%%%%%%%%%%%%%%%%
\section{Wärmestrahlung}

In diesem Versuch wird die Temperatur des Glühdrahtes einer Glühbirne gemessen, unter Ausnutzung der Wärmestrahlung, die von ihm abgegeben wird.

\subsection{Grundlagen, Theorie und Versuchsaufbau}

Jeder Körper emittiert Wärmestrahlung. Ab einer gewissen Temperatur kommt auch sichtbares Licht dazu (wie eben in einer Glühbirne).\\

\textbf{Aufbau:}\\
Die Glühlampe, ein Filter und eine Fotozelle (zur Messung der Intentsität) sind auf einer Schiene montiert. Strom sowie Spannung, mit der die Glühbirne versorgt wird, werden mit 2 Multimetern gemessen (um letztendlich die Leistung zu berechnen); ein weiteres Voltmeter misst die Spannung, nachdem ein Strom-Spannungs-Wandler den Output der Fotozelle übersetzt hat (Abb. \ref{fig:waermestrahlung_aufbau}).




\end{multicols}
\begin{figure}[H]
	\centering
	\includegraphics[scale=0.9]{./data/waermestrahlung_aufbau.png}
	\caption{Versuchsaufbau zur Messung der Wärmestrahlung einer Glühlampe}
	\label{fig:waermestrahlung_aufbau}
\end{figure}
\begin{multicols}{2}


Das Licht wird gefiltert um monochromatisches bzw. einen sehr eng eingegrenzten Wellenlängenbereich zu messen (hier $\lambda = 560nm)$. In diesem Versuch wird nämlich das Planck'sche Strahlungsgesetz verwendet, das die Strahlungsenergie, abhängig von Temperatur des strahlenden Körpers und Wellenlänge der Strahlung, angibt:
$$L(\lambda,T)\cdot d \lambda = \frac{2hc^2}{\lambda ^5}\cdot \frac{1}{exp(\frac{hc}{\lambda k T})-1}\cdot d\lambda$$
Verglichen werden im Versuch 2 vorgegebene Einstellungen an der Glühbirne; der Filter bleibt gleich und damit auch die Wellenlänge; Im Quotienten aus beiden Strahlungsformeln ergibt sich, bei gleicher Wellenlänge, dass die Energien im Verhältnis der Intensitäten stehen:
$$\frac{L(\lambda, T_1)}{L(\lambda, T_2)}=\frac{I_1}{I_2}=\frac{exp(\frac{ch}{k\lambda T_2})}{exp(\frac{ch}{k \lambda T_1})}$$

Durch logarithmieren erhält man:

$$ln(\frac{I_1}{I_2})=(\frac{1}{T_2}-\frac{1}{T_1})\cdot \frac{ch}{k\lambda}$$

Weiters wird das Stefan-Boltzmann-Gesetz von der Umwandlung von elektrischer Leistung in Wärme verwendet, um (wiederum im Quotienten beider Konfigurationen) eine Temperatur zu substituieren:

$$\frac{P_1}{P_2}=\frac{T_1^4}{T_2^4}$$

Dieses Verhältnis wird auch benutzt, um, nach Berechnung einer Temperatur, die zweite zu erhalten.\\
Eingesetzt ergibt sich zur Berechnung von $T_1$:
$$T_1=\frac{1}{ln(\frac{I_1}{I_2})}\cdot (\sqrt[4]{\frac{P_1}{P_2}}-1)\cdot \frac{ch}{k\lambda}$$


Die Leistungen $P_1$ und $P_2$ werden durch die gemssenen Ströme und Spannungen berechnet, 
$\lambda$ durch den verwendeten Filter definiert.\\
Um den Quotienten aus den Intensitäten zu erhalten, wird benutzt, dass die Beleuchtungsstärke der Fotozelle durch den Abstand von der Lichtquelle mit $1/r^2$ verringert.\\

Es werden also, für beide Einstellungen an der Lichtquelle, jeweils mehrere Beleuchtungsstärken mit der Fotozelle, für verschiedenen Abständen von der Lichtquelle, gemessen. Diese werden gegen $1/r^2$ aufgetragen, um einen linearen Anstieg zu ergeben. Im linearen Fit beider Daten-Sets erhält man 2 Steigungen, die im gleichen Verhältnis stehen, wie die Intensitäten:
$$\frac{I_1}{I_2}=\frac{r_1^2}{r_2^2}$$.



\subsection{Resultate}

\subsection{Diskussion}




\section{Quellen}
$[1]$ Anleitung, \url{http://www.univie.ac.at/anfpra/neu1/ps/ps6/PS6.pdf}\\

\end{multicols}

\end{document}