% -*- TeX:de -*-
\NeedsTeXFormat{LaTeX2e}
\documentclass[12pt,a4paper]{article}
\usepackage[german]{babel} % german text
\usepackage[DIV12]{typearea} % size of printable area
\usepackage[T1]{fontenc} % font encoding
%\usepackage[latin1]{inputenc} % most likely on Windows
\usepackage[utf8]{inputenc} % probably on Linux
\usepackage{multicol}

% PLOTTING
\usepackage{pgfplots} 
\usepackage{pgfplotstable}
\usepackage{url}
\usepackage{graphicx} % to include images
\usepackage{tikz}
\usepackage{subfigure} % for creating subfigures
\usepackage{amsmath} % a bunch of symbols
\usepackage{amssymb} % even more symbols
\usepackage{booktabs} % pretty tables
\usepackage{makecell} % multi row table heading

% a floating environment for circuits
\usepackage{float}
\usepackage{caption}

%\newfloat{circuit}{tbph}{circuits}
%\floatname{circuit}{Schaltplan}

% a floating environment for diagrams
%\newfloat{diagram}{tbph}{diagrams}
%\floatname{diagram}{Diagramm}

\selectlanguage{german} % use german

\begin{document}

%%%%%%% DECKBLATT %%%%%%%
\thispagestyle{empty}
			\begin{center}
			\Large{Fakultät für Physik}\\
			\end{center}
\begin{verbatim}


\end{verbatim}
							%Eintrag des Wintersemesters
			\begin{center}
			\textbf{\LARGE SS 14}
			\end{center}
\begin{verbatim}


\end{verbatim}
			\begin{center}
			\textbf{\LARGE{Physikalisches Praktikum\\ für das Bachelorstudium}}
			\end{center}
\begin{verbatim}




\end{verbatim}

			\begin{center}
			\textbf{\LARGE{PROTOKOLL}}
			\end{center}
			
\begin{verbatim}

\end{verbatim}

			\begin{flushleft}
			\textbf{\Large{Experiment (Nr., Titel): PS9 - Heißluftmotor - Stirlingprozess}}\\
							%Experiment Nr. und Titel statt den Punkten eintragen
			\LARGE{PS09 }	
			\end{flushleft}

\begin{verbatim}

\end{verbatim}	
							%Eintragen des Abgabedatums, oder des Erstelldatums des Protokolls
			\begin{flushleft}
			\textbf{\Large{Datum:}} \Large{03.04.2014}
			\end{flushleft}
			
\begin{verbatim}
\end{verbatim}
							%Namen der Protokollschreiber
		\begin{flushleft}
			\textbf{\Large{Namen:}} \Large{Patrick Braun, Johannes Kurz}
			\end{flushleft}

\begin{verbatim}


\end{verbatim}
							%Kurstag und Gruppennummer, zb. Fr/5
			\begin{flushleft}
			\textbf{\Large{Kurstag/Gruppe:}} \Large{DO/4}
			\end{flushleft}

\begin{verbatim}

\end{verbatim}
							%Name des Betreuers, das Praktikum betreute.
			\begin{flushleft}
			\LARGE{\textbf{Betreuer:}}	\Large{Johanna Akbarzadeh}	
			\end{flushleft}

%%%%%%% DECKBLATT ENDE %%%%%%%
\pagebreak
\setlength{\columnsep}{20pt}
\begin{multicols}{2}

%%%%%%%%%%%%%%%%%%%%%%%%%%%%%%%%%%%%%%%%%%%%%%%%

%\begin{figure}[H]
%	\centering
%	\includegraphics[scale=0.35]{./data/beugung.png}
%	\caption{Beugungsmuster Einzelspalt (echtes Foto; schwarz durch weiß ersetzt)}
%	\label{fig:beugungsmuster}
%\end{figure}


%\begin{figure}[H]
%	\centering
%	\pgfplotstabletypeset[
%			columns={abstand, n},
%			col sep=&,
%			columns/abstand/.style={precision=2, zerofill, column name=\makecell{$Abstand$\\$(\pm 0.05)[mm]$} }, 
%			columns/n/.style={column name=\makecell{$n$\\$(Ordnung)$}, precision=0},
%			every head row/.style={before row=\hline,after row=\hline\hline},
%			every last row/.style={after row=\hline},
%			every first column/.style={column type/.add={|}{} },
%			every last column/.style={column type/.add={}{|} }
%			]{
%			abstand & n
%			12.9 & 1
%			24.45 & 2
%			37.40 & 3
%			49.35& 4
%			62.45 & 5
%			74.45 & 6
%			87.45 & 7
%			100.25 & 8
%			
%			}
%	\caption{Messwerte Einzelspalt}
%	\label{tab:werte_einzelspalt}
%\end{figure}


%%%%%%%%%%%%%%%%%%%%%%%%%%%%%%%%%%%%%%%%%%%%%%%%
%%%%%%%%%%%%%%%%%%%%%%%%%%%%%%%%%%%%%%%%%%%%%%%%
\noindent In PS09 



\section{Der Heißluftmotor als Wärmekraftmaschine}
\subsection{Grundlagen}

\subsection{Versuchsaufbau}

Volumen = Weg ($\pm 0.08mm) * Fläche 28.3 + 195$\\
Druck $(-2000 - +2000 \pm 2) hPa$

\subsection{Resultate}

Unbelastet:\\
$A_1 = 37130 hPa*cm^3$
$A_2 = 38140 hPa*cm^3$
$f_2 = \frac{25.32}{3} Hz$
$A_3 = 37530 hPa *cm^3$
$f_3 = \frac{37.82}{5} Hz$
$A_4 = 36660 hPa * cm^3$
$f_4 = \frac{25.28}{3} Hz$
$A_5 = 37000 hPa * cm^3$
$f_5 = \frac{76.18}{10}$

$A_6 = 38720  hPa * cm^3$
$A_7 = 37250  hPa * cm^3$
$f_{6-1} = 7.77 Hz$
$f_{6-2 / 7} = 15.49 / 2 Hz$

Belastet:\\
$r = (25.0 \pm 0.2)cm$
$F_1 = (1 \pm 0.05)N$
$f_1 = 54 / 10Hz$
$f_2 = 55 / 10Hz$


Strom: $(20 \pm 1) A$\\
Spannung: $(14 \pm 0.5) V$\\

\subsection{Diskussion}




%%%%%%%%%%%%%%%%%%%%%%%%%%%%%%%%%%%%%%%%%%%%%%%%
%%%%%%%%%%%%%%%%%%%%%%%%%%%%%%%%%%%%%%%%%%%%%%%%
\section{Die Stirling-Maschine als Kältemaschine}

\subsection{Grundlagen}


\subsection{Resultate}

Zugeführte Leistung:\\
$230V * (0.36 \pm 0.03) A$\\
Temperatur:\\ 
$T_{unten} = (5.2 \pm 0.1)^{\circ}C$\\
$T_{oben} = (6.0 \pm 0.1)^{\circ}C$\\

Kühlung:\\ 
$U_{unten} = (8.0 \pm 0.5)V$
$A_{unten} = (1.7 \pm 0.1)A$
$U_{oben} = (8.5 \pm 0.5)V$\\
$A_{oben} = (1.8 \pm 0.1)A$\\

Frequenz:\\
$f = \frac{50.6}{10} Hz$

\subsection{Diskussion}


\section{Quellen}
$[1]$ Anleitung, \url{http://www.univie.ac.at/anfpra/neu1/ps/ps9/PS9.pdf}\\

\end{multicols}



\end{document}