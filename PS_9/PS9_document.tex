% -*- TeX:de -*-
\NeedsTeXFormat{LaTeX2e}
\documentclass[12pt,a4paper]{article}
\usepackage[german]{babel} % german text
\usepackage[DIV12]{typearea} % size of printable area
\usepackage[T1]{fontenc} % font encoding
%\usepackage[latin1]{inputenc} % most likely on Windows
\usepackage[utf8]{inputenc} % probably on Linux
\usepackage{multicol}

% PLOTTING
\usepackage{pgfplots} 
\usepackage{pgfplotstable}
\usepackage{url}
\usepackage{graphicx} % to include images
\usepackage{tikz}
\usepackage{subfigure} % for creating subfigures
\usepackage{amsmath} % a bunch of symbols
\usepackage{amssymb} % even more symbols
\usepackage{booktabs} % pretty tables
\usepackage{makecell} % multi row table heading

% a floating environment for circuits
\usepackage{float}
\usepackage{caption}

%\newfloat{circuit}{tbph}{circuits}
%\floatname{circuit}{Schaltplan}

% a floating environment for diagrams
%\newfloat{diagram}{tbph}{diagrams}
%\floatname{diagram}{Diagramm}

\selectlanguage{german} % use german

\begin{document}

%%%%%%% DECKBLATT %%%%%%%
\thispagestyle{empty}
			\begin{center}
			\Large{Fakultät für Physik}\\
			\end{center}
\begin{verbatim}


\end{verbatim}
							%Eintrag des Wintersemesters
			\begin{center}
			\textbf{\LARGE SS 14}
			\end{center}
\begin{verbatim}


\end{verbatim}
			\begin{center}
			\textbf{\LARGE{Physikalisches Praktikum\\ für das Bachelorstudium}}
			\end{center}
\begin{verbatim}




\end{verbatim}

			\begin{center}
			\textbf{\LARGE{PROTOKOLL}}
			\end{center}
			
\begin{verbatim}

\end{verbatim}

			\begin{flushleft}
			\textbf{\Large{Experiment (Nr., Titel): PS9 - Heißluftmotor - Stirlingprozess}}\\
							%Experiment Nr. und Titel statt den Punkten eintragen
			\LARGE{PS09 }	
			\end{flushleft}

\begin{verbatim}

\end{verbatim}	
							%Eintragen des Abgabedatums, oder des Erstelldatums des Protokolls
			\begin{flushleft}
			\textbf{\Large{Datum:}} \Large{10.04.2014}
			\end{flushleft}
			
\begin{verbatim}
\end{verbatim}
							%Namen der Protokollschreiber
		\begin{flushleft}
			\textbf{\Large{Namen:}} \Large{Patrick Braun, Johannes Kurz}
			\end{flushleft}

\begin{verbatim}


\end{verbatim}
							%Kurstag und Gruppennummer, zb. Fr/5
			\begin{flushleft}
			\textbf{\Large{Kurstag/Gruppe:}} \Large{DO/4}
			\end{flushleft}

\begin{verbatim}

\end{verbatim}
							%Name des Betreuers, das Praktikum betreute.
			\begin{flushleft}
			\LARGE{\textbf{Betreuer:}}	\Large{Johanna Akbarzadeh}	
			\end{flushleft}

%%%%%%% DECKBLATT ENDE %%%%%%%
\pagebreak
\setlength{\columnsep}{20pt}
\begin{multicols}{2}

%%%%%%%%%%%%%%%%%%%%%%%%%%%%%%%%%%%%%%%%%%%%%%%%

%\begin{figure}[H]
%	\centering
%	\includegraphics[scale=0.35]{./data/beugung.png}
%	\caption{Beugungsmuster Einzelspalt (echtes Foto; schwarz durch weiß ersetzt)}
%	\label{fig:beugungsmuster}
%\end{figure}


%\begin{figure}[H]
%	\centering
%	\pgfplotstabletypeset[
%			columns={abstand, n},
%			col sep=&,
%			columns/abstand/.style={precision=2, zerofill, column name=\makecell{$Abstand$\\$(\pm 0.05)[mm]$} }, 
%			columns/n/.style={column name=\makecell{$n$\\$(Ordnung)$}, precision=0},
%			every head row/.style={before row=\hline,after row=\hline\hline},
%			every last row/.style={after row=\hline},
%			every first column/.style={column type/.add={|}{} },
%			every last column/.style={column type/.add={}{|} }
%			]{
%			abstand & n
%			12.9 & 1
%			24.45 & 2
%			37.40 & 3
%			49.35& 4
%			62.45 & 5
%			74.45 & 6
%			87.45 & 7
%			100.25 & 8
%			
%			}
%	\caption{Messwerte Einzelspalt}
%	\label{tab:werte_einzelspalt}
%\end{figure}


%%%%%%%%%%%%%%%%%%%%%%%%%%%%%%%%%%%%%%%%%%%%%%%%
%%%%%%%%%%%%%%%%%%%%%%%%%%%%%%%%%%%%%%%%%%%%%%%%
\noindent Im Folgenden werden unterschiedliche Anwendungen einer Stirling-Maschine untersucht: als Wärmekraftmaschine auf Grundlage des Carnotprozesses und in abgeänderter Form als Kältemaschine.\\
Als wichtigstes Maß eines Motors wird der Wirkungsgrad jeder Anwendung ermittelt.


\section{Der Heißluftmotor als Wärmekraftmaschine}
In diesem Teil des Experiments bestimmen wir den Wirkungsgrad eines Stirlingmotors über die Leistung welche zugeführt wird und abgenommen werden kann. Zuerst wird der ideale Wirkungsgrad bestimmt, danach der realen Wirkungsgrad und zuletzt der Wirkungsgrad des Motors unter Belastung.

\subsection{Grundlagen}
Für die Berechnungen werden folgende Formeln benötigt ([1] (pp. 6-8)):\\
$$\eta_{ideal} = \frac{T_1 - T_2}{T_1}$$
Bei diesem idealen Verhalten geht keine Energie über Reibung oder Abstrahlung verloren.
$$\eta_{real} = \frac{P_{Motor}}{P_{Zugeführt}}$$
Der reale Wirkungsgrad ist aussagekräftiger: Dabei wird die aufgewendete Leistung der abnehmbaren Leistung gegenüber gestellt. Verluste und Ungenauigkeiten durch reale Abweichungen von den theoretischen Näherungen gehen mit ein.
$$\eta_{Motor} = \frac{P_{Motor}}{\oint p dV * f}$$
Den Wirkungsgrad des Motors erhalten wir durch die Leistung des Motors (jeweilig beschrieben durch Abnehmer) geteilt durch die Leistung des Motors selbst, beschrieben durch die eingeschlossene Fläche im Carnotprozess (entspricht der Energie) mal der Frequenz. 
\\
Im realen Prozess ergibt sich aber keine schön abgegrenzte Fläche wie in [1] dargestellt, sondern eine Form, wie in Abbildung \ref{fig:real_carnot}:

\begin{figure}[H]
	\centering
	\includegraphics[scale=0.25]{./data/kennlinie_stirling_ohnelast.png}
	\caption{Kennlinie im pV-Diagramm eines realen Prozesses im Leerlauf}
	\label{fig:real_carnot}
\end{figure}

\subsection{Versuchsaufbau}
In diesem Experiment wird ein bereits aufgebauter Stirlingmotor mit einer Heizspindel und einer Kühlung betrieben (Abbildung  \ref{fig:stirlingMotor_3D}). Als Grundlage dient der bereits beschriebene Carnotprozess. Druck und Volumen werden mit Cassy-Lab erfasst. Die Kühlung und die Heizung können direkt über Netzteile gesteuert werden.\\
Die Frequenz wird  mit einer Scheibe (mit Löchern im gleichen Radius) und einem Stroboskop gemessen:\\
Die Scheibe wird (magnetisch) am Schwungrad des Motors angebracht. Im Betrieb wird mit dem Stroboskoplicht auf die Scheibe geleuchtet. Entspricht die Frequenz des Stroboskops einem vielfachen der Drehfrequenz der Scheibe, erscheinen die Löcher stehend. Durch eine Markierung auf der Scheibe lässt sich schließlich der Faktor zwischen beiden Frequenzen direkt abzählen.
\\
Vor der Messung muss in Cassy das kleinste und größte Volumen im Zylinder bestimmt und fixiert werden.
Das Volumen des Zylinder ergibt sich durch (auf den Nullpunkt angepasst) als $195cm^3$
%$Volumen = Weg (\pm 0.08mm) * Fläche (28.3 \pm 0.1)mm + 195 = $\\ %%%% TODO ergebnis %%%%%
%Der Bereich für den Druck\\
%$Druck (-2000 - +2000 \pm 2) hPa$\\
Anschließend kann der Motor, bei laufender Heizung, angeworfen werden und die Messungen durchgeführt werden.\\

Die Messung des Wirkungsgrades unter Belastung wird mit Hilfe eines Bremszaums durchgeführt: Die Vorrichtung aus 2 Stangen, die verbunden mit Schrauben, bei einem einstellbaren Druck auf die Achse diese bremsen, wird an ihrem Ende an einer Federwaage befestigt.\\
Die daran verrichtete Leistung ist gegeben durch
$$P=\vec{M}\cdot \vec{\omega}$$
$$\vec{M}=\vec{F}\times \vec{r}$$
Daher müssen die Schrauben so eingestellt werden, dass der Bremszaum im rechten Winkel zur Federwaage steht, um mit Beträgen rechnen zu können. Der Radius der Bremsvorrichtung wird mit dem Maßband bestimmt, $\omega$ mit der Stroboskop-Methode und die Kraft kann direkt von der Federwaage abgelesen werden.

\end {multicols}{2}
\begin{figure}[H]
	\centering

	\includegraphics[scale=0.45]{./data/3D-Model/PS9-model_neutral01.JPG}

	\caption{Versuchsaufbau als 3D Modell}
	\label{fig:stirlingMotor_3D}
\end{figure}

\begin{multicols}{2}

\subsection{Resultate}
\textbf{1. Leerlaufbetrieb:}\\
Leerlaufkreisfrequenz:\\
$\omega=(50.0 \pm 1.3)s^{-1}$
$$\eta_{ideal}=(54.6 \pm 2.8)\%$$
\textbf{2. Unbelastet:}\\
Heizleistung:\\
Strom: $(20 \pm 1) A$\\
Spannung: $(14 \pm 0.5) V$\\
$P_{Heizung}=(280 \pm 18)W$\\
$$\eta_{real}=(10.65 \pm 0.71)\%$$\\

%
%$A_1 = 37130 hPa*cm^3$\\
%$A_2 = 38140 hPa*cm^3$\\
%$f_2 = \frac{25.32}{3} Hz$\\
%$A_3 = 37530 hPa *cm^3$\\
%$f_3 = \frac{37.82}{5} Hz$\\
%$A_4 = 36660 hPa * cm^3$\\
%$f_4 = \frac{25.28}{3} Hz$\\
%$A_5 = 37000 hPa * cm^3$\\
%$f_5 = \frac{76.18}{10}$\\
%
%
%
%$A_6 = 38720  hPa * cm^3$\\
%$A_7 = 37250  hPa * cm^3$\\
%$f_{6-1} = 7.77 Hz$\\
%$f_{6-2 / 7} = 15.49 / 2 Hz$\\



\noindent \textbf{3. Belastet:}\\
$r = (25.0 \pm 0.2)cm$\\
$F_1 = (1.00 \pm 0.05)N$\\
$f_1 = (5.4 \pm 0.05)Hz$\\
$\eta_{motor1}=(3.03\pm 0.25)\%$\\
$F_2=(0.97\pm 0.05)N$\\
$f_2 = (5.5 \pm 0.05)Hz$\\
$\eta_{motor2}=(2.99\pm 0.25)\%$\\
$$\eta_{motor}=(3.01\pm 0.25)\%$$


\subsection{Diskussion}

Die Unterschiede zwischen idealem und realem Wirkungsgrad, sowie demjenigen unter Belastung sind deutlich zu erkennen:\\
$\eta_{ideal}=(54.6 \pm 2.8)\%$\\
$\eta_{real}=(10.65 \pm 0.71)\%$\\
$\eta_{motor}=(3.01\pm 0.25)\%$\\

\noindent Die Verluste durchReibung, Konvektion und Wärmestrahlung sowie die unrealistische Annahme eines vollständig reversiblen Prozesses führen zu einem etwa 5mal niedrigeren realem Wirkungsgrad im Leerlauf, im Vergleich zur Theorie.\\
Die Reibungsverluste bei Belastung, also die eines Motors, der effektiv Arbeit an einer Vorrichtung verrichtet, reduzieren den Wirkungsgrad nochmals um 1/3.\\

\noindent Der vergleichsweise große Unsicherheitsbereich von $\eta_{ideal}$ entsteht dabei durch eine größere Fehlerabschätzung bei Druck und Volumen, als das Messvermögen der CASSY-Sensoren verlangen würde. Da jedoch der aufgezeichnete Kreisprozess nicht komplett stabil (im Rahmen der Auflösung) läuft, haben wir uns für einen etwas gröberen Bereich entschieden.\\

\noindent Die Messergebnisse zeigen das erwartete Verhalten und demonstrieren eindrucksvoll die schwache Umsetzung des Stirling-Motors von Energie in nutzbare Arbeit.\\
Sicherlich gäbe es jedoch einiges Verbesserungsmöglichkeiten; Auch ohne den Motor fundamental umzubauen, ließen sich vor allem die beweglichen Teile sicherlich noch besser lagern um Reibungsverluste zu minimieren.\\
Andere Materialien im Bereich, in dem Temperaturaustausch stattfindet könnten hier zu besseren Ergebnissen führen.\\
Ein Getriebe oder auch ein optimiertes Schwungrad könnten zumindest den Wirkungsgrad im Vollbetrieb verbessern.\\
Große Verbesserungen sind jedoch nicht zu erwarten.\\

\noindent Die Messung der Frequenz mit Stroboskop und Lochscheibe scheint, abgesehen von komplizierteren Sensor-Messaufbauten, die beste Lösung für schnelle und genaue Ergebnisse zu sein. Auch wenn völliger Stillstand der wahrgenommenen Löcher nicht zu erreichen war, sind die Ergebnisse jedoch mehr als genau genug im Vergleich zu den Schwankungen des Motors selbst.\\
Der reale Wirkungsgrad wurde aus einem Mittel aus 6 Einzelmessungen (pV-Diagramm sowie Frequenz) im Abstand von etwa 1min bestimmt. Die Unterschiede zwischen den einzelnen Messungen sind dabei deutlich größer, als die Messunsicherheit der Stroboskop-Methode.\\
Auch die Messung mit dem Bremszaum ist im Vergleich zu den Schwankungen des Motors selbst, ausreichend genau (ideale $90^\circ$ zwischen Bremszaum und Federwaage sind natürlich nicht gegeben, der Winkel bleibt jedoch gut stabil und die Abweichung des Winkels hat nur kleine Auswirkungen auf das Ergebnis).



%%%%%%%%%%%%%%%%%%%%%%%%%%%%%%%%%%%%%%%%%%%%%%%%
%%%%%%%%%%%%%%%%%%%%%%%%%%%%%%%%%%%%%%%%%%%%%%%%
\section{Die Stirling-Maschine als Kältemaschine}
In diesem Teil des Experiments wird Versucht den Stirlingmotor mit einem externen Motor zu betreiben um Wärme zu transportieren und so einen kühlenden Effekt zu erzeugen.

\subsection{Grundlagen}
Die Grundlagen entsprechen wie im ersten Versuch der Nutzung des Carnotprozesses. In diesem Fall wird aber mechanische Arbeit aufgewendet um vom Wärmereservoir Energie zur Senke zu transportieren. Es findet dabei eine Kühlung statt, da in realen Prozessen keine idealen (unerschöpflichen) Wärmereservoire existieren.\\
Der Wirkungsgrad ist ebenfalls über die allgemeine Definition zu erhalten. Die zugeführte Leistung des Motors kann mit Strom und Spannung errechnet werden. Die Leistungszufuhr für die Heizspindel (welche das Reservoir darstellt) entspricht bei konstant gehaltener Temperatur der (maximalen) Leistungsaufnahme.\\
$$\eta_{Kaelte} = \frac{P_{Wendel}}{P_{Motor}}$$
\subsection{Resultate}


Kühlraumtemperatur:\\ 
$T= (5.6 \pm 0.5)^{\circ}C$\\
\indent (deutliche Schwankungen)\\
Zugeführte Leistung:\\
$P=(82.8 \pm 4.6)W$\\
%$230V * (0.36 \pm 0.03) A$\\
\noindent Kühlung (= Heizleistung der Heizwendel):\\ 
$U= (8.3 \pm 0.7)V$\\
$I = (1.7 \pm 0.2)A$\\
%$U_{oben} = (8.5 \pm 0.5)V$\\
%$A_{oben} = (1.8 \pm 0.1)A$\\
\noindent Umdrehungsfrequenz:\\
$f = (5.06\pm 0.05)Hz$\\
Wirkungsgrad:\\
$$\eta_{Kaelte}=(17.04\pm 2.7)\%$$

\subsection{Diskussion}

Die Kältemaschine ist die Umkehrung der Wärmekraftmaschine. Da der Antrieb (ein Elektromotor) nun mit einem Wirkungsgrad von $100\%$ angenommen wird und, abgesehen von den Kolben in der Maschine selbst, keine mechanischen Teile bewegt werden, ist der Wirkungsgrad der Kältemaschine deutlich höher, als der des Wärme-Motors.\\
Dabei werden in diesem Versuch einige Annahmen getroffen:\\
\begin{itemize}
\item Elektromotor mit idealem Wirkungsgrad
\item Heizleistung = Kühlleistung
\item ideale Wärmeisolierung
\item stabile Gleichgewichtstemperatur
\end{itemize}

\noindent Ein schlechterer Wirkungsgrad des E-Motors würde einen real höheren Wirkungsgrad der Kältemaschine bedeuten. Damit wirkt er den anderen Punkten entgegen.\\
In Wirklichkeit hat sich keine stabile Gleichgewichtstemperatur eingestellt; Die Heizleistung wurde so gesteuert, dass sich eine Oszillation zwischen 2 Temperaturen ($\Delta T \backsim 1^\circ C$) eingestellt hat.\\
An beiden Extrempunkten wurde gemessen und der Wirkungsgrad berechnet. Aufgrund der zugrunde liegenden Gleichungen ergibt sich im Mittel beider Wirkungsgrade der gleiche Wert, wie für das Mittel der Spannungen/Ströme in der Heizwendel.\\
Aus Gründen der Übersichtlichkeit werden also die gemittelten Messergebnisse für Spannung und Strom der Heizleistung abgedruckt. Daraus ergeben sich die Unsicherheitsbereiche, die deutlich höher sind, als die Messgeräte verlangen würden.
Der Wirkungsgrad und seine Unsicherheit bleiben dabei erhalten.\\
Die Fluktuationen im Ablauf der Kältemaschine, machen die Messung zusätzlich undeutlich. Insgesamt ergibt sich also ein eher hoher Unsicherheitsbereich im Wirkungsgrad der Kältemaschine.\\

\noindent Aufgrund der schlechten Isolierung, arbeitet auch die Raumtemperatur gegen die Kältemaschine. Das angestrebte Gleichgewicht liegt bei etwa $6^\circ C$, also deutlich niedriger als Raumtemperatur. Da die Leistung an der Heizwendel die Messgröße ist, wird der Wirkungsgrad so nach oben verfälscht.\\
Dieser Effekt ließe sich verringern durch ein Angleichen der Umgebungstemperatur an die Zieltemperatur (also durch einen kälteren Raum oder eine höhere Gleichgewichtstemperatur).\\
Dazu müsste jedoch die Kältemaschine etwas anders realisiert sein, was möglicherweise den PS9-Teil mit der Wärmekraftmaschine stören würde bzw. ein aufwendigeres Heiz/Kühlsystem verlangen würde.\\







\section{Quellen}
$[1]$ Anleitung, \url{http://www.univie.ac.at/anfpra/neu1/ps/ps9/PS9.pdf}\\

\end{multicols}

\begin{figure}[H]
	\centering
	\includegraphics[scale=2]{./data/3D-Model/PS9-model_desk01.JPG}
	%\caption{}
	\label{fig:stirlingMotor_3D-desktop}
\end{figure}

\end{document}