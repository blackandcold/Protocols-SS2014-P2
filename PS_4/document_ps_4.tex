% -*- TeX:de -*-
\NeedsTeXFormat{LaTeX2e}
\documentclass[12pt,a4paper]{article}
\usepackage[german]{babel} % german text
\usepackage[DIV12]{typearea} % size of printable area
\usepackage[T1]{fontenc} % font encoding
%\usepackage[latin1]{inputenc} % most likely on Windows
\usepackage[utf8]{inputenc} % probably on Linux
\usepackage{multicol}

% PLOTTING
\usepackage{pgfplots} 
\usepackage{pgfplotstable}
\usepackage{url}
\usepackage{graphicx} % to include images
\usepackage{tikz}
\usepackage{subfigure} % for creating subfigures
\usepackage{amsmath} % a bunch of symbols
\usepackage{amssymb} % even more symbols
\usepackage{booktabs} % pretty tables
\usepackage{makecell} % multi row table heading

% a floating environment for circuits
\usepackage{float}
\usepackage{caption}

%\newfloat{circuit}{tbph}{circuits}
%\floatname{circuit}{Schaltplan}

% a floating environment for diagrams
%\newfloat{diagram}{tbph}{diagrams}
%\floatname{diagram}{Diagramm}

\selectlanguage{german} % use german

\begin{document}

%%%%%%% DECKBLATT %%%%%%%
\thispagestyle{empty}
			\begin{center}
			\Large{Fakultät für Physik}\\
			\end{center}
\begin{verbatim}


\end{verbatim}
							%Eintrag des Wintersemesters
			\begin{center}
			\textbf{\LARGE SS 14}
			\end{center}
\begin{verbatim}


\end{verbatim}
			\begin{center}
			\textbf{\LARGE{Physikalisches Praktikum\\ für das Bachelorstudium}}
			\end{center}
\begin{verbatim}




\end{verbatim}

			\begin{center}
			\textbf{\LARGE{PROTOKOLL}}
			\end{center}
			
\begin{verbatim}

\end{verbatim}

			\begin{flushleft}
			\textbf{\Large{Experiment (Nr., Titel):}}\\
							%Experiment Nr. und Titel statt den Punkten eintragen
			\LARGE{PS04 Spektroskopie und Interferometrie}	
			\end{flushleft}

\begin{verbatim}

\end{verbatim}	
							%Eintragen des Abgabedatums, oder des Erstelldatums des Protokolls
			\begin{flushleft}
			\textbf{\Large{Datum:}} \Large{6.3.2014}
			\end{flushleft}
			
\begin{verbatim}
\end{verbatim}
							%Namen der Protokollschreiber
		\begin{flushleft}
			\textbf{\Large{Namen:}} \Large{Patrick Braun, Johannes Kurz}
			\end{flushleft}

\begin{verbatim}


\end{verbatim}
							%Kurstag und Gruppennummer, zb. Fr/5
			\begin{flushleft}
			\textbf{\Large{Kurstag/Gruppe:}} \Large{DO/4}
			\end{flushleft}

\begin{verbatim}

\end{verbatim}
							%Name des Betreuers, das Praktikum betreute.
			\begin{flushleft}
			\LARGE{\textbf{Betreuer:}}	\Large{Schafler}	
			\end{flushleft}

%%%%%%% DECKBLATT ENDE %%%%%%%
\pagebreak
\setlength{\columnsep}{20pt}
\begin{multicols}{2}

%%%%%%%%%%%%%%%%%%%%%%%%%%%%%%%%%%%%%%%%%%%%%%%%

%\begin{figure}[H]
%	\centering
%	\includegraphics[scale=0.35]{./figure/beugung.png}
%	\caption{Beugungsmuster Einzelspalt (echtes Foto; schwarz durch weiß ersetzt)}
%	\label{fig:beugungsmuster}
%\end{figure}


%\begin{figure}[H]
%	\centering
%	\pgfplotstabletypeset[
%			columns={abstand, n},
%			col sep=&,
%			columns/abstand/.style={precision=2, zerofill, column name=\makecell{$Abstand$\\$(\pm 0.05)[mm]$} }, 
%			columns/n/.style={column name=\makecell{$n$\\$(Ordnung)$}, precision=0},
%			every head row/.style={before row=\hline,after row=\hline\hline},
%			every last row/.style={after row=\hline},
%			every first column/.style={column type/.add={|}{} },
%			every last column/.style={column type/.add={}{|} }
%			]{
%			abstand & n
%			12.9 & 1
%			24.45 & 2
%			37.40 & 3
%			49.35& 4
%			62.45 & 5
%			74.45 & 6
%			87.45 & 7
%			100.25 & 8
%			
%			}
%	\caption{Messwerte Einzelspalt}
%	\label{tab:werte_einzelspalt}
%\end{figure}



%%%%%%%%%%%%%%%%%%%%%%%%%%%%%%%%%%%%%%%%%%%%%%%%
%%%%%%%%%%%%%%%%%%%%%%%%%%%%%%%%%%%%%%%%%%%%%%%%
\section{Grundlagen, Theorie und Versuchsaufbau}

\subsection{Auflösungsvermögen eines Gitters}
%To DO: Abbildung

In diesem Versuch geht es darum, das Auflösungsvermögen eines Gitters für ein bekanntes Lichtwellenspektrum zu erforschen, und aus dem Setup mit veränderlicher effektiver Gitterbreite die Gitterkonstante $a$ zu ermitteln.\\
In PW8 des Anfängerpraktikums I wurden erstmals die Lichtbeugung und Interferenz am Doppelspalt sowie die Beugung am Gitter untersucht. Durch Messung der unterschiedlichen Beugungswinkel verschiedener scharf abgegrenzter Wellenlängen, wurde aus dem gemessenen Spektrum auf die Art der Lichtquelle, eine Quecksilberdampflampe, geschlossen.\\
Der Aufbau mit Lichtquelle, Fernrohr, Gitter und Beobachtungsfernrohr an einem Goniometer wird auch nun benutzt, um die unbekannte Gitterkonstante zu bestimmen.\\


%%%%% ABBILDUNG DES VERSUCHSAUFBAU %%%%%%


Ausgehend vom Doppelspalt, wird die Ausprägung der Interferenzmaxima bei steigender Spaltanzahl stärker. Werden die Beugungsmaxima am Gitter einzelner Wellenlängen der Hg-Lampe also beobachtet, zeigen sich klar erkennbare deutlich abgegrenzte Linien in den entsprechenden Farben.\\
Verringert man jedoch die Spaltanzahl, indem eine Blende mit variabler Breite vor das Gitter gebracht wird, beginnen die Grenzen der Spektrallinien mit enger werdendem Blendenspalt, zu verschmieren.\\
\\
Da, auch für eine hohe Spaltanzahl ($N \rightarrow \infty $), die Intensitätsmaxima nicht unendlich scharf werden, müssen 2 unterschiedliche Wellenlängen einen gewissen Abstand haben (abhängig von N) um unterscheidbar zu sein (in einem kontinuierlichen Spektrum ist das eben nicht der Fall).\\
Dieses Auflösungsvermögen $A$ ist gegeben durch
$$A= \frac{\lambda}{\Delta \lambda} = n \cdot N$$
wobei $\Delta \lambda$ die Differenz der beiden Wellenlängen ist, und $n$ die Ordnung des Interferenzmaximums.\\
\\
In diesem Versuch wird benutzt, dass das Hg-Spektrum 2 sehr knapp nebeneinanderliegende Wellenlängen im gelben Bereich hat. Sind diese bekannt, lässt sich $A$ berechnen und durch die Ordnungszahl des betrachteten Maximums auch die Spaltanzahl $N$.\\
Die effektiv wirkende Bereich des Gitters ist begrenzt durch die Blendenbreite $B$, deren Betrag gleich N mal der Gitterkonstante a ist: 
$$N=\frac {A}{n}= \frac{\lambda}{\Delta \lambda \cdot n}$$
$$B = a \cdot N$$
$$\Rightarrow a = \frac {B \cdot n}{A}$$

Die Wellenlängen der beiden gelben Linien sind bekannt, aus ihnen wird zunächst A berechnet.\\
Danach wird jeweils ein Beugungsmaximum dieser Doppellinie ausgewählt und die Blende vor dem Gitter so lange verengt, bis die beiden Linien nicht mehr unterscheidbar sind. Es ist also die Grenze des Auflösungsvermögens erreicht.\\
Die Breite der Blende wird anschließend mit einem Kathetometer bestimmt: Das ist ein Fernrohr mit Fadenkreuz, dass durch eine Schraube parallel verschoben werden kann. Zuerst wird eine Seite des Spalts fokussiert, dann an der Schraube gedreht, bis das Fadenkreuz auf die andere Seite zielt. Auf der Skala der Schraube ist die gemessene Breite ablesbar.\\
\\
Die Messung wurde für 3 Ordnungen der Beugungsmaxima jeweils auf beiden Seiten der 0-ten Ordnung gemessen.\\




\subsection{Spektrometrie}


\subsection{Michelson-Interferometer}



%\end{multicols}
%\begin{multicols}{2}



%%%%%%%%%%%%%%%%%%%%%%%%%%%%%%%%%%%%%%%%%%%%%%%%
%%%%%%%%%%%%%%%%%%%%%%%%%%%%%%%%%%%%%%%%%%%%%%%%
\section{Resultate}
\subsection{Auflösungsvermögen eines Gitters}


\begin{figure}[H]
	\centering
	\pgfplotstabletypeset[
			columns={abstand_L, abstand_R, n},
			col sep=&,
			columns/abstand_L/.style={precision=2, zerofill, column name=\makecell{Spaltbreite L\\$(\pm 0.01)$\\$[mm]$} }, 
			columns/abstand_R/.style={precision=2, zerofill, column name=\makecell{Spaltbreite R\\$(\pm 0.01)$\\$[mm]$} }, 
			columns/n/.style={column name=\makecell{$n$\\$(Ordnung)$}},
			every head row/.style={before row=\hline,after row=\hline\hline},
			every last row/.style={after row=\hline},
			every first column/.style={column type/.add={|}{} },
			every last column/.style={column type/.add={}{|} }
			]{
			abstand_L & abstand_R & n
			1.24 & 1.52& 2
			0.79 & 0.87& 3
			0.51 & 0.66 & 4
						
			}
	\caption{Messwerte der Spaltbreite der Blende für linke und rechte Maxima}
	\label{tab:blendenbreite}
\end{figure}



\textbf{Einzelergebnisse:}\\
links:\\
2.Odg.: $a_{2l}=(9.05\pm 0.37) \mu m$\\
3. Odg.: $a_{3l}=(8.65 \pm 0.55) \mu m$\\
4. Odg.: $a_{4l}=(7.45\pm 0.74)\mu m$\\
rechts:\\
2. Odg.: $a_{2r}=(8.62\pm 0.37)\mu m$\\
3. Odg.: $a_{3r}=(9.53\pm 0.55) \mu m$\\
4. Odg.:$a_{4r}=(9.64 \pm 0.74) \mu m$\\
\\
discarded:\\
2.Odg r: $a_{2r-disc.}=(11.10 \pm 0.37)\mu m$\\
\\
\textbf{Mittelwert:}\\
$$a=(8.82 \pm 0.33)\mu m$$




\subsection{Spektrometrie}


\subsection{Michelson-Interferometer}


%%%%%%%%%%%%%%%%%%%%%%%%%%%%%%%%%%%%%%%%%%%%%%%%
%%%%%%%%%%%%%%%%%%%%%%%%%%%%%%%%%%%%%%%%%%%%%%%%
\section{Diskussion}
\subsection{Auflösungsvermögen eines Gitters}

Die Auflösung des Kathetometers beträgt $0.01$ mm, die Unsicherheit für die Messungen der Blendenbreiten wurde jedoch mit $0.05$ mm abgeschätzt. Da sich in Versuchen, die gleiche Breite mehrmals zu messen, aufgrund des Fadenkreuzes und der ausgefransten Spaltenränder, eine gewisse Toleranz der Ergebnisse eingestellt hat, scheint diese eher vorsichtige Abschätzung vernünftig.\\
Würde das Messinstrument öfter verwendet und ein besseres Gefühl für seine Handhabung entwickelt werden, wäre es sicher realistisch, kleinere unsicherheiten (eventuell bis hin zur Auflösung) zuzulassen.\\
\\
Im Kapitel \emph{Resultate} sind die einzelnen Ergebnisse für die Gitterkonstante, an jedem betrachteten Interferenzmaxiumum angegeben, sowie ein gemittelter Wert über alle Ergebnisse.\\
Die Gitterkonstante des verwendeten Gitters wurde also mit \\
$a=(8.82 \pm 0.33)\mu m$\\
bestimmt.\\
Da jedoch die Grenze des Auflösungsvermögens der beiden gelben Spektrallinien in einem gewissen Bereich subjektiv ist, und auch von verschiedenen Umfeldfaktoren abhängt, darf ein wesentlich größerer Unsicherheitsbereich angenommen werden, als der, der sich aus den Messunsicherheiten ergibt.\\
Eine Einschränkung ist die Intensität der betrachteten Linien. Diese ist in der 4. Ordnung ohnehin schon sehr klein, außerdem treten hier die engsten Blendeneinstellungen auf. Dies verringert die Intensität zusätzlich und lässt die Messung sehr schwierig werden. Das zeigt sich auch darin, dass beide Messungen an den Maxima 4. Ordnung die extremsten Ergebnisse darstellen.\\
Aber auch die anderen Messungen sind von momentanen Einflüssen gestört: Da die Blende jeweils abgenommen werden muss, um die Blendenbreite zu bestimmen, wird das Gitter immer wieder bewegt (die Blende wird fest an das Gitter geklemmt). Eine feste Vorrichtung zum Einspannen würde möglicherweise Verbesserungen bringen.\\
Außerdem wurde das Experiment in einem Raum mit einer zweiten Gruppe durchgeführt, die etwas Licht für ihre Arbeit benötigt haben. Auch das ist ein Störeinfluss, der zwar leicht zu beseitigen wäre, aber hier aus praktischen Gründen vorliegt.\\
Es lässt sich also nicht mit Bestimmtheit sagen, dass der wahre Wert innerhalb des Bereiches liegt.\\Nachfragen ergab, dass die Lösung mit $10 \mu m$ angegeben ist. Da jedoch alle Einzelmessungen unter diesem Wert liegen, ist trotz aller Ungenauigkeiten ein systematischer Fehler wahrscheinlich (bzw. eine ungenaue Angabe von Herstellerseite). Dies wäre in erster Näherung zu überprüfen, indem man die Ergebnisse verschiedener Gruppen miteinander vergleicht.\\
\\
Ein einziges Ergebnis liegt über $10\mu m$. Dieses wurde jedoch, vor seiner Berechnung verworfen, da der Messende am Goniometer während der Kathetometermessung unterbrochen hat mit der Begründung, wegen tränender Augen, die Messung am Maximum rechts, 2. Ordnung, wiederholen zu wollen.\\



\subsection{Spektrometrie}


\subsection{Michelson-Interferometer}


\section{Quellen}
$[1]$ Anleitung, \url{http://www.univie.ac.at/anfpra/neu1/ps/ps4/ps4.pdf}\\
\end{multicols}

\end{document}